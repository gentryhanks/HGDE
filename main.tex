\documentclass[12pt]{article}
\usepackage[english]{babel}
\usepackage[utf8x]{inputenc}
\usepackage{amsmath}
\usepackage{graphicx}
\usepackage{setspace} 
\usepackage{natbib}

\title{Historical Geographies of Diabetes and Emotion}
\author{gENTRY hANKS}
\begin{document}
\maketitle


%\newpage
%\begin{abstract}
%\singlespacing 
%\end{abstract}

\newpage
\section{Background}
\doublespacing
Mainstream historical accounts of diabetes are well documented \citep{engelhardt_diabetes_1989, tattersall_diabetes_2009}. In 1000 BCE, Susruta, an Indian physician, discovered diabetes, but the condition didn't receive this name until Greek physician, Aretaeus in 100 BCE. He used the Greek word, \textit {dia-bainein} meaning ``to siphon'' \citep{sattley_history_1996}.  

In the 17th century the term \textit{Mellitus}, the Latin for ``honeyed,'' was tacked on to \textit{Diabetes}, giving us the contemporarily used term \textit{Diabetes Mellitus}, by a physician from London, Thomas Wills \citep{sattley_history_1996}. Wills arrived at this term through sampling his patients' urine, which, if it tasted sweet like honey, meant a diagnosis of Diabetes Mellitus. The tasting of urine\footnote{Urine and blood are key bodily fluids for surveillance in the management of diabetes. PWD are required to take a snapshot as evidence of blood glucose level at a particular time with a blood glucose meter.} remained the standard for monitoring glucose levels into the 1900s \citep{sattley_history_1996}. 

Physicians were all but left to watch their patients fade away. Many prescribed low-calorie diets, but little else prolonged the lives of people with diabetes until the discovery of human-consumable insulin \citep{ebstein_history_1989}. Of course this discovery didn't come without the help of companion animals \citep{balfe_diabetes_2008}\footnote{See \citet{wilkie_multispecies_2013} for research on health and multi-species encounters.} and agricultural livestock. In 1921, Canadian surgeon, Banting, along with Best, treated a canine with diabetes by injecting extracts from a non-diabetic dog's pancreas \citep{zimmermann_first_1989}. From there they joined Drs. Collip and Macleod \footnote{The Best and Banting Collections in the Fisher Rare Book Library at the University of Toronto reveals a contested claim and ownership over the patent right of insulin between Best and Banting and Macleod and Collip.} in injecting a purer form of animal insulin into an adolescent, Leonard Thompson, whose high blood sugar lowered over the next 24 hours \citep{sattley_history_1996}.  

In 1935, Hinsworth delineated something that had been understood as one illness into two types \citep{sattley_history_1996}. There are people with insulin sensitivity, but without the capability to produce insulin (Type I) and others with insensitivity, but with the capability to produce insulin (Type II).  With this breakthrough research in diabetes proliferated bringing with it medical and technological innovation. Towards the end of the 1930s various kinds of beef and pork insulin were developed to try and match the speed and variance of human insulin. While the livestock based insulin was a tremendous help, insulin therapy was nowhere near as effective as a human pancreas \citep{sattley_history_1996}. During the disovery of insulin in Canada, Joslin was the first doctor in the US to do comparative and complementary research on insulin therapy.

\citet{daniels_role_1948}, a doctor in the field of psychosomatic medicine in the 1930s and 1940s explained that the lack of attention to the role of emotion in diabetes sprang from a lack of evidence supporting a relationship between war stress and an appreciable increase in diabetes cases in post-WWI soldiers:

\begin{singlespace}
  \begin{quote}
  At this time, Joslin\dots entirely reversed an earlier tentative position that emotion might have a part in the onset and course of diabetes and issued an authoritative statement to the contrary. Chief emphasis was laid on obesity and heredity, with a complete denial that emotional factors may even significantly influence the sugar level during the course of the disease (p. 288).
  \end{quote} 
\end{singlespace} 

This particular change of focus has greatly influenced the geneticization and biomedicalization of diabetes. Daniels's call for an attention to the role of emotion in the onset and course of diabetes was in effect silenced by Joslin, one of the most well known names in the diabetes medical community \footnote{Joslin is still a well known name in diabetes medical research because his research legacy remains visible in clinics dedicated to diabetes research and patient treatment at the main clinic in Boston and branches throughout the US.}. Daniels wasn't convinced: 

\begin{singlespace}
  \begin{quote} 
	A counter-current to the receding tide of medical interest in emotional factors in diabetes appeared in the reexamination of the literature and direct observation of clinical cases by psychoanalytically-oriented psychiatrists in 1935—36. Both the reevaluation of literature and the case material demonstrated unequivocally the role of emotion in the course of the disease by influencing the blood-sugar level in established diabetes. This has been further amply confirmed. Observations point in certain cases to a correlation between depression or conversion symptoms and increased sugar, and also between exhilaration and anxiety symptoms and a temporary clearing of or decrease in sugar(p. 288).
  \end{quote} 
\end{singlespace}

Over the next several decades there was a proliferation of synthetic insulins, oral medications, syringes, urine test strips, glucose meters, insulin pumps, and other new technologies \citep{phillip_attd_2012} for the treatment and management of diabetes. The drive in innovation has generally been to make these apparatuses smaller and more portable to enhance mobility, which consequently increased one-time-use (disposable) supplies\footnote{My initial analysis of the data reveals a sense of guilt associated with the amount of biomedical waste and its disposal for some PWD.}. With the arrival of the insulin pump and an artificial pancreas we continue on a post-human trajectory\footnote{See \citet{wilson_more_2011} for a reflection on mobility, digital frontiers and more-than-human contact.}.  

\subsection{How Is This Geographical?}
Firstly I bring to the fore an overview of different perspectives on diabetes from different places and pieces of history. Secondly after my examination of medical literature, historical writings and archival material I put forth that, over time and place, different geographical locations within the body have come to be understood as the ``seat'' of the illness. And finally, I will include historical examples of what role emotion plays and how it has been understood throughout a history of diabetes. 


\section{The Eradication of Emotion} 
\doublespacing
The role of emotion has been seen as more prevalent and important, particularly in the first half of the twentieth century. The bulk of research and writing arrived through psychosomatic medicine. In the late 1800s and early 1900s, there was a distinction made based on the etiology of one's diabetes. The initial presentation of diabetes after prolonged times of sorrow, anxiety and crisis was classified as emotional glycosuria. Emotional glycosuria also referred to increased sugar levels in the urine of those who already have diabetes following cases of mental illness and depression. Although medical doctors in the past and present acknowledge that emotion plays a role in the course of the illness, exactly how and to what degree has been and still is not well understood. Astutely, \citet{burch_1962_role} noted, ``that as new understanding of the disturbed physiology of the disease has developed, or as new advances have been made in therapy, interest in the role of emotional factors has receded'' (p, 131/93). While the main current focus of medical communities centers on heredity and obesity, a focus on emotion has largely fallen by the wayside, particularly in medical fields that have achieved legitimacy through their willingness to neglect the role of emotion in human health. This has created a rift in the treatment of diabetes-- maintaining a split between mind and body-- and has been positioned as a metabolic disorder. This mind/body split in current medical practice relies on the assumption that emotion is not bodily and vice versa. 
This split is furthered through a carving up of geopgraphical dilineations of the body, almost competely obscuring the concept that the mind/body dualism is a false one. The carving up of bodies, as it were, paralells that of medical disciplines and academic fields in general. 

The current focus on heredity and obesity in the discourse surrounding diabetes hasn't always had the lime light. Emotion was once very much considered as part of the etiology of diabetes. While we acknowledge that food is a major factor in diabetes, we neglect the emotional and cultural connections to food as agents of belonging and identity. Historically the prime way of treating diabetes invovled a restricted diet, which in conjunction with polyuria, lead to dangerously low body weights. Throughout the history of diabetes research body size has certainly taken up its fair share of ink, paper and computer screens, but is in most instances deployed to reify the notion that obesity is the main culprit of Type 2 diabetes and that people with Type 1 diabetes should be or are typically thin (add archival letter from dr to Banting about his female T1D patient struggling with weight gain). The many cases of people with Type 2 being thin and cases of people with T1D as larger are severely overlooked. 

Contemporarily, it is rare that a physician takes into account the emotional factors in the course of diabetes (among other illnesses). There has been a turn in North America to `fix' this with bandaids called diabetic educators. The current model of treatment relies on the individual requisitioning a team of doctors and professionals, thereby splitting one's own person into compartments based solely on the bodily geographic location of symptoms or secondary problems. This team often consists of a family doctor, an endocrinologist, an opthamologist, a nutritionist or dietician, a podiatrist, and a gynocologist (for women). Oddly, although men's sexual and reproductive health is also affected by diabetes, it is almost unheard of that they are approached about these topics outside of written information plastered on walls and layed out on waiting room tables, let alone are men encouraged to broach this subject with medical professionals. 

While there seemed to be a trend toward understanding causal and correlational relationships between emotion and diabetes, this trend faded with the rise of a focus on obesity, medicalization and genetics. Only now in and after the affective turn do we again see a rise in interest between the two. 

Interestingly in the last hundred years we have seen people desperate for insulin therapy (as seen in the letters to Drs. Banting and Best) and have come full circle to a phenomenon called diabulima, whereby one restricts insulin intake in order to lose weight or to maintain a lower weight. Much like Anorexia Nervosa or Bulimia, receiving compliments on one's weight or looks after practicing diabulima only serves as a positive reinforcement to continue underuse of insulin. Likewise the ability to eat almost anything and not gain weight, as well as not having to pay for insulin and use needles to inject it makes diabulimia all the more appealing. 


\citet{boehm_1878_beitrage} experimented on cats whereby they observed glucose levels in the urine after exposure to several conditions. It was later found that physical pain, bondage and temperature weren't necessary ingredients for raising levels of sugar in the urine, but although Boehm and Hoffman didn't acknowledge it in their publication, emotional excitement was certainly involved.  

\section{Annotations and Quotes}
Medical and academic literature regarding diabetes produced from the 1930s through the 1970s is saturated with snippets of biological and environmental determinism, which ultimately allowed the baby to be thrown out with the bath water.  
\begin{enumerate}
\item \citep{daniels_role_1948}
``In seriously considering emotional conflict in the etiology, it is not necessary to discard facts relating either to heredity or obesity, as both appear of great clinical importance and must be included in any calculation'' (p. 289). 
\item \citep{cannon_1916_bodily}
When Bohm and Hoffman's experiment was repeated to address the emotional factors, which they had not addressed in their results other than to intimate that the designation of ``Fesselungsdiabetes'' was not justifiable as ``emotional glycosuria.'' Their results found that pain was the contributing factor in elevated sugar levels in the cats. The discovery that ``during fright (or rage?) the adrenal sectretion is increased, and the fact that injection of epinephrin gives rise to glycosuria, suggested taht glycosuria might be called forth by emotional excitement'' (p. 282). When the experiment was repeated without the element of pain, an increase in sugar in the urine occurred. 

\item \citep{bond_1895_relation}
Bond referred to diabetes as a ``physician-baffling disease'' (p. 295).
Dr. Goodall (1895) doesn't believe insanity to be related to insanity; however, he states ``Although perhaps diabetes does not go with insanity direct, yet persons suffering from diabetes undoubtedly show
various morbid psychical manifestations. They are neurotic in many ways; members of neurotic families no doubt ; they show hypochondriasis, irritability, sometimes excitement, mania, and so on. They have hysterical manifestations and mental instability'' (p. 311). 

\item \citep{menninger_1935_psychological} Menninger conducted a thorough review of pre 1934 literature dealing with emotion and raised sugar levels in the urine and blood. 

\item \citep{major_1933_classic}
One of Hippocrates disciples, Aretaeus, is attributed with the first use of the word \textit diabetes in connection with a description of symptoms associated with diabetes. This is generally accepted, but not without criticism. According to


\item \citep{sanders_2001_philatelic}
Sanders warns the reader at the beginning of his book that there is no way to provide a complete or whole history and ``the omission of any event or individual's role in the history of diabetes in no way lessens the importance of that contribution'' (p. xiii). 
\citet{sanders_2001_philatelic} names the 4 divisions of the history of diabetes, ``The Descriptive Period: describing and naming the disease, The Diagnostic Period: learning how to diagnose the disease, The Experimental Period: learning what causes the disease and the Therapeutic Era: learning how to treat the disease.'' (p. 1), which are well accepted by medical hisotrians \citep{papaspyros_1964_history}. Sanders has also offered a fifth period, ``The Era of Complications, in which we learn how diabetes causes additional health problems'' (p. 1). 
\end{enumerate}


\newpage
\section{Major Players}
It might be of use to introduce researchers and doctors who have been particularly visible in the world of diabetes focused medicine as well as historians of diabetes. 
Thomas Willis
G. E. Daniels
Papaspyros
Charles Best
Frederick Banting
Joslin
George Burch
Collip
McLeod
White, P. 

%\subsection{Lists}

\newpage
\singlespacing
\bibliographystyle{apa}
\bibliography{dissertation_bibliography}

\end{document}