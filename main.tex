\documentclass[12pt]{article}
\usepackage[english]{babel}
\usepackage[utf8x]{inputenc}
\usepackage{amsmath}
\usepackage{graphicx}
\usepackage{setspace} 
\usepackage{natbib}

\title{Historical Geographies of Diabetes and Emotion}
\author{gENTRY hANKS}
\begin{document}
\maketitle


\newpage
\begin{abstract}
\singlespacing 
The current focus on heredity and obesity in the discourse surrounding diabetes hasn't always had the lime light. Emotion was once very much considered as part of the etiology of diabetes. 

\end{abstract}

\newpage
\section{Major Players}
It might be of use to introduce researchers and doctors who have been particularly visible in the world of diabetes focused medicine as well as historians of diabetes. 
Thomas Willis
G. E. Daniels
Papaspyros
Charles Best
Frederick Banting
Joslin
George Burch

\section{The Eradication of Emotion} 
\doublespacing
The role of emotion has been seen as more prevalent and important, particularly in the 1950s, 60s and 70s. The bulk of research and writing arrived through psychosomatic medicine. Although medical doctors in the past and present acknowledge that emotion plays a role in the course of the illness, exactly how and to what degree has been hotly contested. Astutely, \citet{burch_1962_role} noted, ``that as new understanding of the disturbed physiology of the disease has developed, or as new advances have been made in therapy, interest in the role of emotional factors has receded'' (p, 131/93). While the main current focus of medical communities centers on heredity and obesity, a focus on emotion has largely fallen by the wayside, particularly in medical fields that have achieved legitimacy through their willingness to neglect the role of emotion in human health. This has created a rift in the treatment of diabetes-- maintaining a split between mind and body-- and has been positioned as a metabolic disorder. 
It is rare that a physician takes into account the emotional factors in the course of diabetes (among other illnesses). There has been a turn to fix this with bandaids called diabetic educators. The current model of treatment relies on the individual requisitioning a team of doctors and professionals, thereby splitting a person into compartmentalized medical problems based solely on the bodily geographic location of symptoms or secondary problems. This team often consists of a family doctor, an endocrinologist, an opthamologist, a nutritionist or dietician, a podiatrist, and a gynocologist (for women). Oddly, although men's sexual and reproductive health is also affected by diabetes, it is almost unheard of that men are approached about these topics outside of written information, let alone encouraged to bring it up at a doctor's visit. 
%\subsection{Lists}

\newpage
\singlespacing
\bibliographystyle{apa}
\bibliography{dissertation_bibliography}

\end{document}