\documentclass[12pt]{article}
\usepackage[english]{babel}
\usepackage[utf8x]{inputenc}
\usepackage{amsmath}
\usepackage{graphicx}
\usepackage{setspace} 
\usepackage{natbib}

\title{Historical Geographies of Diabetes and Emotion}
\author{gENTRY hANKS}
\begin{document}
\maketitle


\newpage
\begin{abstract}
\doublespacing 
The current focus on heredity and obesity in the discourse surrounding diabetes hasn't always had the lime light. Emotion was once very much considered as part of the etiology of diabetes. 

\end{abstract}

\newpage
\section{Major Players}
It might be of use to introduce researchers and doctors who have been particularly visible in the world of diabetes focused medicine as well as historians of diabetes. 
Thomas Willis
G. E. Daniels
Papaspyros
Charles Best
Frederick Banting
Joslin
George Burch

\section{The Eradication of Emotion} 

The role of emotion has been seen as more prevalent and important, particularly in the 1950s, 60s and 70s. The bulk of research and writing arrived through psychosomatic medicine. Although medical doctors in the past and present acknowledge that emotion plays a role in the course of the illness, exactly how and to what degree has been hotly contested.  
%\subsection{Lists}

\newpage
\singlespacing
\bibliographystyle{apa}
\bibliography{dissertation_bibliography}

\end{document}